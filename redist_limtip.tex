% Options for packages loaded elsewhere
\PassOptionsToPackage{unicode}{hyperref}
\PassOptionsToPackage{hyphens}{url}
\PassOptionsToPackage{dvipsnames,svgnames,x11names}{xcolor}
%
\documentclass[
  11pt,
]{article}

\usepackage{amsmath,amssymb}
\usepackage{iftex}
\ifPDFTeX
  \usepackage[T1]{fontenc}
  \usepackage[utf8]{inputenc}
  \usepackage{textcomp} % provide euro and other symbols
\else % if luatex or xetex
  \usepackage{unicode-math}
  \defaultfontfeatures{Scale=MatchLowercase}
  \defaultfontfeatures[\rmfamily]{Ligatures=TeX,Scale=1}
\fi
\usepackage{lmodern}
\ifPDFTeX\else  
    % xetex/luatex font selection
\fi
% Use upquote if available, for straight quotes in verbatim environments
\IfFileExists{upquote.sty}{\usepackage{upquote}}{}
\IfFileExists{microtype.sty}{% use microtype if available
  \usepackage[]{microtype}
  \UseMicrotypeSet[protrusion]{basicmath} % disable protrusion for tt fonts
}{}
\makeatletter
\@ifundefined{KOMAClassName}{% if non-KOMA class
  \IfFileExists{parskip.sty}{%
    \usepackage{parskip}
  }{% else
    \setlength{\parindent}{0pt}
    \setlength{\parskip}{6pt plus 2pt minus 1pt}}
}{% if KOMA class
  \KOMAoptions{parskip=half}}
\makeatother
\usepackage{xcolor}
\usepackage[lmargin=1in,rmargin=1in,tmargin=1in,bmargin=1in]{geometry}
\setlength{\emergencystretch}{3em} % prevent overfull lines
\setcounter{secnumdepth}{3}
% Make \paragraph and \subparagraph free-standing
\ifx\paragraph\undefined\else
  \let\oldparagraph\paragraph
  \renewcommand{\paragraph}[1]{\oldparagraph{#1}\mbox{}}
\fi
\ifx\subparagraph\undefined\else
  \let\oldsubparagraph\subparagraph
  \renewcommand{\subparagraph}[1]{\oldsubparagraph{#1}\mbox{}}
\fi


\providecommand{\tightlist}{%
  \setlength{\itemsep}{0pt}\setlength{\parskip}{0pt}}\usepackage{longtable,booktabs,array}
\usepackage{calc} % for calculating minipage widths
% Correct order of tables after \paragraph or \subparagraph
\usepackage{etoolbox}
\makeatletter
\patchcmd\longtable{\par}{\if@noskipsec\mbox{}\fi\par}{}{}
\makeatother
% Allow footnotes in longtable head/foot
\IfFileExists{footnotehyper.sty}{\usepackage{footnotehyper}}{\usepackage{footnote}}
\makesavenoteenv{longtable}
\usepackage{graphicx}
\makeatletter
\def\maxwidth{\ifdim\Gin@nat@width>\linewidth\linewidth\else\Gin@nat@width\fi}
\def\maxheight{\ifdim\Gin@nat@height>\textheight\textheight\else\Gin@nat@height\fi}
\makeatother
% Scale images if necessary, so that they will not overflow the page
% margins by default, and it is still possible to overwrite the defaults
% using explicit options in \includegraphics[width, height, ...]{}
\setkeys{Gin}{width=\maxwidth,height=\maxheight,keepaspectratio}
% Set default figure placement to htbp
\makeatletter
\def\fps@figure{htbp}
\makeatother

\makeatletter
\@ifpackageloaded{float}{}{\usepackage{float}}
\floatstyle{plain}
\@ifundefined{c@chapter}{\newfloat{atbl}{h}{loatbl}}{\newfloat{atbl}{h}{loatbl}[chapter]}
\floatname{atbl}{Table A}
\floatstyle{plaintop}
\restylefloat{atbl}
\newcommand*\quartoatblref[1]{Table \hyperref[#1]{A\ref{#1}}}
\@ifpackageloaded{caption}{}{\usepackage{caption}}
\DeclareCaptionLabelFormat{quartoatblreflabelformat}{#1#2}
\captionsetup[atbl]{labelformat=quartoatblreflabelformat}
\newcommand*\listofatbls{\listof{atbl}{List of Appendix Tabless}}
\makeatother
\makeatletter
\@ifpackageloaded{caption}{}{\usepackage{caption}}
\AtBeginDocument{%
\ifdefined\contentsname
  \renewcommand*\contentsname{Table of contents}
\else
  \newcommand\contentsname{Table of contents}
\fi
\ifdefined\listfigurename
  \renewcommand*\listfigurename{List of Figures}
\else
  \newcommand\listfigurename{List of Figures}
\fi
\ifdefined\listtablename
  \renewcommand*\listtablename{List of Tables}
\else
  \newcommand\listtablename{List of Tables}
\fi
\ifdefined\figurename
  \renewcommand*\figurename{Figure}
\else
  \newcommand\figurename{Figure}
\fi
\ifdefined\tablename
  \renewcommand*\tablename{Table}
\else
  \newcommand\tablename{Table}
\fi
}
\@ifpackageloaded{float}{}{\usepackage{float}}
\floatstyle{ruled}
\@ifundefined{c@chapter}{\newfloat{codelisting}{h}{lop}}{\newfloat{codelisting}{h}{lop}[chapter]}
\floatname{codelisting}{Listing}
\newcommand*\listoflistings{\listof{codelisting}{List of Listings}}
\makeatother
\makeatletter
\makeatother
\makeatletter
\@ifpackageloaded{caption}{}{\usepackage{caption}}
\@ifpackageloaded{subcaption}{}{\usepackage{subcaption}}
\makeatother
\ifLuaTeX
  \usepackage{selnolig}  % disable illegal ligatures
\fi
\usepackage{bookmark}

\IfFileExists{xurl.sty}{\usepackage{xurl}}{} % add URL line breaks if available
\urlstyle{same} % disable monospaced font for URLs
\hypersetup{
  pdftitle={The Hidden Poor: Solving Time Poverty through Redistribution of Household Production},
  pdfauthor={Fernando Rios-Avila; Aashima Sinha},
  pdfkeywords={Time Poverty, Income Poverty, Redistribution , household
production, care work, gender equality, LIMTIP},
  colorlinks=true,
  linkcolor={blue},
  filecolor={Maroon},
  citecolor={Blue},
  urlcolor={Blue},
  pdfcreator={LaTeX via pandoc}}


\usepackage{datetime}
\usepackage{booktabs}
\usepackage{chngcntr}
\usepackage{apptools}
\AtAppendix{\counterwithin{table}{section}}
\AtAppendix{\counterwithin{figure}{section}}

\title{The Hidden Poor: Solving Time Poverty through Redistribution of
Household Production}
\author{
Fernando Rios-Avila\\
Levy Economics Institute\\
\\
\and 
Aashima Sinha\\
Levy Economics Institute\\
\\
}
\date{2024-05-09}
\begin{document}


\def\spacingset#1{\renewcommand{\baselinestretch}%
{#1}\small\normalsize} \spacingset{1}

%Ipsum lorem

\maketitle
\begin{abstract}
xxx
\end{abstract}
 
\vspace{.2in}

\textbf{\textit{Keyword: }}
    Time Poverty, Income Poverty, Redistribution , household production,
care work, gender equality, LIMTIP, 
    Time Poverty, Income Poverty, Redistribution , household production,
care work, gender equality, LIMTIP 


\thispagestyle{empty}
\clearpage\pagenumbering{arabic}
\newpage
\spacingset{1.2} % DON'T change the spacing!
\section{Introduction}\label{introduction}

Redistribution of household production has been identified as an
important tool to achieve gender equality (Elson 1995; xx())he The
incorporation of the 3R (recognizition, reduction adn redistribution)
strategy as a target in the sustainable development goals, is a
testament to the decades of activism and advocacy emphasizing that
inequality on this front is not purely or even primarily a ``private
family matter'' but a matter of public policy. While redistribution of
household production responsibilities from females to males is important
intrsinsically for human rights and fairness concerns, it is also
instrumental in achieving gender equality in labor market outcomes
(Bruyn-Hundt 1996; Elson 2017; Esquivel 2016). Yet, difficult questions
remain about public policies and collective actions that would reduce
inequality, especially in poorer countries. A limited consensus seems to
have emerged regarding the effectiveness of certain policy initiatives
(e.g., paid paternity leave). But, many of them are likely to have only
limited efficacy in the poorer countries due to their structural
features such as the widespread absence of formal wage labor and weak
welfare states.

In the case of the US,

\section{LIMTIP: A New Measure of Time Poverty for the United
States}\label{limtip-a-new-measure-of-time-poverty-for-the-united-states}

\begin{itemize}
\tightlist
\item
  Describe the LIMTIP measure and how it is constructed: Methods paper
\item
  Brief description of the LIMTIP measure and the Hidden Poor in the US.
  Small section
\end{itemize}

Using ATUS and ASEC data and utilzing statitical matching we develop
income and time poverty estimates for the United States for the years
2005 to 2022. In this policy brief we focus on discussing the limptip
estimates for the year 2022. Further, we develop three redistribution
scenarios wherein we alter the share of household production among
household memeber and examine if limtip estiamtes change for individuals
and housheolds.

\section{Identifying the Problem}\label{identifying-the-problem}

\begin{itemize}
\tightlist
\item
  The problem we need to identify the problem of time-poverty caused by
  redistribution (or lack thereof) of household production.
\item
  Identify either: how many Time poor individuals live in household with
  time non-poor adults.
\item
  Or identify the baseline of time poverty if there is full flexibility
  for time allocation. (Household Deficit consider both time deficits
  and surpluses)
\end{itemize}

This would give us a fist look at how much poverty could be alleviated
if household production was redistributed.

We could even look at Who are this individuals who are living in time
poverty, but that Do not need to. (describe the characteristics of these
individuals)

\begin{itemize}
\item
  Added value. This will help us identify those who cannot be helped by
  redistribution of household production. (even if their incidence
  changes
\item
  This raises the question. Do we want to analyze redistribution in
  household that are not time poor?
\end{itemize}

\section{Redistribution Scenarios}\label{redistribution-scenarios}

\begin{itemize}
\item
  Here we would describe the three redistribution scenarios we have
  developed. This would be ``realistic'' scenarios.
\item
  Describe the scenarios and the assumptions behind them.
\end{itemize}

\section{Results}\label{results}

\begin{itemize}
\item
  Compare the time poverty changes on those identified earlier.
\item
  Moreover, we examine the changes by sex, employment status, presence
  of children in the household.
\item
  Perhaps Start with a global analysis (without Specific groups)
\item
  then Analyze the case for Specific groups
\end{itemize}

Perhaps provide more emphasis on Some of the groups (gender, the
employed, parents?)

\subsection{Gender Disparity in the Incidence of Time
Deficits}\label{gender-disparity-in-the-incidence-of-time-deficits}

Evidence indicates that negative time balance values (i.e., time
deficits) occur mostly among employed persons Add table by sex and
employment status for limtip and discuss results. Further, presence of
children would demand more caregiving hours thereby putting a pressure
on time available, particulalrly for employed couples or single memebr
hosuehodls. Add table by sex and children for limtip and discuss
results.

\#\#Gender and race Disparity in the Incidence of Time Deficits

\section{Policy implications}\label{policy-implications}

\section{Conclusion}\label{conclusion}



\end{document}
