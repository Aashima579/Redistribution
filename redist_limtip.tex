% Options for packages loaded elsewhere
\PassOptionsToPackage{unicode}{hyperref}
\PassOptionsToPackage{hyphens}{url}
\PassOptionsToPackage{dvipsnames,svgnames,x11names}{xcolor}
%
\documentclass[
  11pt,
]{article}

\usepackage{amsmath,amssymb}
\usepackage{iftex}
\ifPDFTeX
  \usepackage[T1]{fontenc}
  \usepackage[utf8]{inputenc}
  \usepackage{textcomp} % provide euro and other symbols
\else % if luatex or xetex
  \usepackage{unicode-math}
  \defaultfontfeatures{Scale=MatchLowercase}
  \defaultfontfeatures[\rmfamily]{Ligatures=TeX,Scale=1}
\fi
\usepackage{lmodern}
\ifPDFTeX\else  
    % xetex/luatex font selection
\fi
% Use upquote if available, for straight quotes in verbatim environments
\IfFileExists{upquote.sty}{\usepackage{upquote}}{}
\IfFileExists{microtype.sty}{% use microtype if available
  \usepackage[]{microtype}
  \UseMicrotypeSet[protrusion]{basicmath} % disable protrusion for tt fonts
}{}
\makeatletter
\@ifundefined{KOMAClassName}{% if non-KOMA class
  \IfFileExists{parskip.sty}{%
    \usepackage{parskip}
  }{% else
    \setlength{\parindent}{0pt}
    \setlength{\parskip}{6pt plus 2pt minus 1pt}}
}{% if KOMA class
  \KOMAoptions{parskip=half}}
\makeatother
\usepackage{xcolor}
\usepackage[lmargin=1in,rmargin=1in,tmargin=1in,bmargin=1in]{geometry}
\setlength{\emergencystretch}{3em} % prevent overfull lines
\setcounter{secnumdepth}{3}
% Make \paragraph and \subparagraph free-standing
\ifx\paragraph\undefined\else
  \let\oldparagraph\paragraph
  \renewcommand{\paragraph}[1]{\oldparagraph{#1}\mbox{}}
\fi
\ifx\subparagraph\undefined\else
  \let\oldsubparagraph\subparagraph
  \renewcommand{\subparagraph}[1]{\oldsubparagraph{#1}\mbox{}}
\fi


\providecommand{\tightlist}{%
  \setlength{\itemsep}{0pt}\setlength{\parskip}{0pt}}\usepackage{longtable,booktabs,array}
\usepackage{calc} % for calculating minipage widths
% Correct order of tables after \paragraph or \subparagraph
\usepackage{etoolbox}
\makeatletter
\patchcmd\longtable{\par}{\if@noskipsec\mbox{}\fi\par}{}{}
\makeatother
% Allow footnotes in longtable head/foot
\IfFileExists{footnotehyper.sty}{\usepackage{footnotehyper}}{\usepackage{footnote}}
\makesavenoteenv{longtable}
\usepackage{graphicx}
\makeatletter
\def\maxwidth{\ifdim\Gin@nat@width>\linewidth\linewidth\else\Gin@nat@width\fi}
\def\maxheight{\ifdim\Gin@nat@height>\textheight\textheight\else\Gin@nat@height\fi}
\makeatother
% Scale images if necessary, so that they will not overflow the page
% margins by default, and it is still possible to overwrite the defaults
% using explicit options in \includegraphics[width, height, ...]{}
\setkeys{Gin}{width=\maxwidth,height=\maxheight,keepaspectratio}
% Set default figure placement to htbp
\makeatletter
\def\fps@figure{htbp}
\makeatother

\makeatletter
\@ifpackageloaded{float}{}{\usepackage{float}}
\floatstyle{plain}
\@ifundefined{c@chapter}{\newfloat{atbl}{h}{loatbl}}{\newfloat{atbl}{h}{loatbl}[chapter]}
\floatname{atbl}{Table A}
\floatstyle{plaintop}
\restylefloat{atbl}
\newcommand*\quartoatblref[1]{Table \hyperref[#1]{A\ref{#1}}}
\@ifpackageloaded{caption}{}{\usepackage{caption}}
\DeclareCaptionLabelFormat{quartoatblreflabelformat}{#1#2}
\captionsetup[atbl]{labelformat=quartoatblreflabelformat}
\newcommand*\listofatbls{\listof{atbl}{List of Appendix Tabless}}
\makeatother
\makeatletter
\@ifpackageloaded{caption}{}{\usepackage{caption}}
\AtBeginDocument{%
\ifdefined\contentsname
  \renewcommand*\contentsname{Table of contents}
\else
  \newcommand\contentsname{Table of contents}
\fi
\ifdefined\listfigurename
  \renewcommand*\listfigurename{List of Figures}
\else
  \newcommand\listfigurename{List of Figures}
\fi
\ifdefined\listtablename
  \renewcommand*\listtablename{List of Tables}
\else
  \newcommand\listtablename{List of Tables}
\fi
\ifdefined\figurename
  \renewcommand*\figurename{Figure}
\else
  \newcommand\figurename{Figure}
\fi
\ifdefined\tablename
  \renewcommand*\tablename{Table}
\else
  \newcommand\tablename{Table}
\fi
}
\@ifpackageloaded{float}{}{\usepackage{float}}
\floatstyle{ruled}
\@ifundefined{c@chapter}{\newfloat{codelisting}{h}{lop}}{\newfloat{codelisting}{h}{lop}[chapter]}
\floatname{codelisting}{Listing}
\newcommand*\listoflistings{\listof{codelisting}{List of Listings}}
\makeatother
\makeatletter
\makeatother
\makeatletter
\@ifpackageloaded{caption}{}{\usepackage{caption}}
\@ifpackageloaded{subcaption}{}{\usepackage{subcaption}}
\makeatother
\ifLuaTeX
  \usepackage{selnolig}  % disable illegal ligatures
\fi
\usepackage{bookmark}

\IfFileExists{xurl.sty}{\usepackage{xurl}}{} % add URL line breaks if available
\urlstyle{same} % disable monospaced font for URLs
\hypersetup{
  pdftitle={The Hidden Poor: Solving Time Poverty through Redistribution of Household Production},
  pdfauthor={Fernando Rios-Avila; Aashima Sinha},
  pdfkeywords={Time Poverty, Income Poverty, Redistribution , household
production, care work, gender equality, LIMTIP},
  colorlinks=true,
  linkcolor={blue},
  filecolor={Maroon},
  citecolor={Blue},
  urlcolor={Blue},
  pdfcreator={LaTeX via pandoc}}


\usepackage{datetime}
\usepackage{booktabs}
\usepackage{chngcntr}
\usepackage{apptools}
\AtAppendix{\counterwithin{table}{section}}
\AtAppendix{\counterwithin{figure}{section}}

\title{The Hidden Poor: Solving Time Poverty through Redistribution of
Household Production}
\author{
Fernando Rios-Avila\\
Levy Economics Institute\\
\\
\and 
Aashima Sinha\\
Levy Economics Institute\\
\\
}
\date{2024-06-03}
\begin{document}


\def\spacingset#1{\renewcommand{\baselinestretch}%
{#1}\small\normalsize} \spacingset{1}

%Ipsum lorem

\maketitle
\begin{abstract}
xxx
\end{abstract}
 
\vspace{.2in}

\textbf{\textit{Keyword: }}
    Time Poverty, Income Poverty, Redistribution , household production,
care work, gender equality, LIMTIP, 
    Time Poverty, Income Poverty, Redistribution , household production,
care work, gender equality, LIMTIP 


\thispagestyle{empty}
\clearpage\pagenumbering{arabic}
\newpage
\spacingset{1.2} % DON'T change the spacing!
\# Introduction

Redistribution of household production has been identified as an
important tool to achieve gender equality (Elson The incorporation of
the 3R (recognizition, reduction adn redistribution) strategy as a
target in the sustainable development goals, is a testament to the
decades of activism and advocacy emphasizing that inequality on this
front cannot be justified in the name of ``private family matter''
rather is a matter of public policy. While redistribution of household
production responsibilities from females to males is important
intrsinsically for human rights and fairness concerns, it is also
instrumental in achieving gender equality in labor market outcomes
(Bruyn-Hundt 1996; Elson 2017; Esquivel 2016). Yet, difficult questions
remain about public policies and collective actions that would reduce
inequality, especially in poorer countries. A limited consensus seems to
have emerged regarding the effectiveness of certain policy initiatives
(e.g., paid paternity leave). But, many of them are likely to have only
limited efficacy in the poorer countries due to their structural
features such as the widespread absence of formal wage labor and weak
welfare states. In the case of the US, issues related to lack of public
prpovisioning of care infrastructure and services, persistence of
childcare deserts, lack of paid parental leave laws among others have
gained momentum. In 2021, the value of unpaid care work in the U.S.
amounted to \$600 billion, constituting approximately 2.6\% of the GDP
(Reinhard et al.~2023). Moreover, like most other countries, we observe
gender disparity in sharing of household work such that Women
disproportionately shoulder the burden of unpaid care work. According to
the 2018 American Time Use Survey, among adults aged 15 and older, women
on average spent 5.7 hours per day on unpaid household and care work,
compared with 3.6 hours for men. In other words, women spent 37 percent
more time on unpaid household and care work than men (Hess \& Hayes
2020). Additionally, the U.S. falls behind many OECD countries in
childcare policies and outcomes, spending only 0.4\% of GDP on early
childhood education and care (ECEC), compared to the OECD average of
0.8\% (OECD 2020). Notably, the U.S. lacks federal laws granting paid
parental leave, setting it apart from other OECD nations. Around 51\% of
the U.S. population resides in childcare deserts, defined as census
tracts with more than 50 children under the age of 5 and either no
childcare providers or significantly limited options, resulting in a
severe shortage of licensed child care slots (Malik et al.~(2018)). In
the above setting, care demand falls onto the household, partculalrly
women. This in turn restricts care providers allocation of time to other
activities including employment, leisure, socializing and self care.
Time-trade off are crucial determines individual's well-being. While
some hosuheolds may be able to outsource some of these care needs, other
income-constraint hosuehodls may not be able to. In the last two
decades, there has been growing interest in studying time and income
poverty and in developing their linkages (xx).

Time constraints that stem from the overlapping domains of paid and
unpaid work are of central concern to the debates surrounding economic
development and gender equality. In this backdrop, we develop a novel
measure of poverty for the U.S. that incorporates time deficits, known
as the Levy Institute Measure of Time and Income Poverty (LIMTIP). Time
deficits due to household production is gaining attention in the U.S
given the persisting lack of publicly provided care, affordable child
and elderly care, and limited paid parental leave benefits. These time
deficits constrain people's time allocation in a range of activities, in
turn affecting their overall well-being, productivity, labor market
participation, and earnings. The consequences are particularly serious
for women due to the disproportionate burden of household
responsibilities they bear, which are closely intertwined with labor
market and well-being outcomes. Standard measures of poverty fail to
capture hardships caused by time deficits and thereby do not provide a
complete picture for effective poverty-alleviation and welfare programs.
Understand the incidence of time poverty that individuals face and how
that may have implications for the study of poverty, gender equality and
overall development is therefore crucial.

\section{LIMTIP: A New Measure of Time Poverty for the United
States}\label{limtip-a-new-measure-of-time-poverty-for-the-united-states}

\begin{itemize}
\tightlist
\item
  Describe the LIMTIP measure and how it is constructed: Methods paper
\item
  Brief description of the LIMTIP measure and the Hidden Poor in the US.
  Small section
\end{itemize}

Poverty is a multidimensional concept that goes beyond the simple notion
of lack of income. In addition to income, poverty can be understood as a
lack of access to resources, including time. LIMTIP is a metric that
incorporates in addition to income poverty, aspects of time poverty.
Time poverty refers to a scenarion wherein people may not have any time
left after engaging in activities that are essential for taking care of
the household, its members, self-care, and paid work (when appropriate).

As with any other measures of poverty, it is necessary to identify a
threshold to determine if given the resources available to a person or
household, they should be classified as poor or non-poor. In the case of
time, however, thinking about a threshold is less intuitive because all
individuals have the same amount of time available to them i.e 24 hours
in a day. Instead, the approach used for the construction of the LIMTIP
has been to identify the time balance people would potentially face
after considering the necessary time spent on essential activities and
household responsabilities. In this framework, people with a negative
balance are considered time-poor. We express the weekly time balance of
individual \(i\) in household \(j\), \(X_{ij}\), as:

\begin{equation}\phantomsection\label{eq-bal}{X_{ij} = 168 - M - \alpha_{ij}R_j-D_{ij}^0(L_{ij}+T_{ij})
}\end{equation}

where 168 is the number of hours in a week, \(M\) is the sum of personal
care and non-substitutable household production requirements, \(R_j\) is
the required time of household production that a family \(j\) needs to
maintain the household, \(\alpha_{ij}\) is the share of individual \(i\)
in the household production requirements. To account for required time
due to working, the Equation~\ref{eq-bal} also includes \(D_{ij}\), a
dummy variable that takes a value of 1 if the person is employed and
zero otherwise. Thus, for those employed, one also accounts for hours of
employment \(L_{ij}\) , the hours of commuting \(T_{ij}\).

To construct this measure, we need a dataset that contains information
on individuals' time use, in addition to standard information required
for poverty analysis. The main source of information for time use comes
from the American Time Use Survey (ATUS), which only provides
information for a single person in the household and a single day. It is
necessary to combine the ATUS with the ASEC data to construct a
synthetic dataset that contains information on time use for all
household members, which will allow us to impute all required variables
for Equation~\ref{eq-bal}.

Using ATUS and ASEC data and utilzing statitical matching we develop
income and time poverty estimates for the United States for the years
2005 to 2022. In this policy brief we focus on discussing the LIMTIP
estimates for the year 2022(!TBD).

The LIMTIP is finally measured as:
\begin{equation}\phantomsection\label{eq-limtip}{P_{j}^L* = P_{j}^O- p_{j}*X_{j}
}\end{equation}

The underlying idea is that if we were to monetize the time deficits of
individuals and add those to the offical income poverty thresholds, we
get an adjusted poverty line - LIMTIP. The estimation of LIMTIP helps us
calculate the number of hidden poor, i.e time poor indivuals who are
left outside the scope of official income poverty estimates. The
difference between the LIMTIP measure of poverty and the official
poverty estimates give us the number of hidden poor.

In (\textbf{fig\_trend?}) we present time trend from 2005 to 2023 of
time poverty estimates, the offical poverty trend, the LIMTIP poverty
trend. We observe that the official poverty estimates shows a slight
rising trend between 2005 to 2014 and then starts to decline. The
pandemic years show some steep decline from 13.8 \% in 2020 to 10\% in
2022 before rising back to the pre-pandemic level of around 15\%. When
we adjust for time deficits, as expected our LIMTIP estimates shows a
higher level of poverty, around 25 percentage points higher . The gap
widened during pandemic years. Moreover, it is notable that time poverty
peaked around Great recession and Covid-19 pandemic recession. While
time poverty rate has remained more or less stabke until 2019, it fell
slightly in 2020 before rising again between 2020 to 2023. We also
observe that in 2022, nearly 33 percent of individuals are hidden poor
in 2022. Further, in Table xx and yy we present the distribution of
individuals and hh respectively across the four-way classification of
LIMTIP (time poor and i) for the year 2022.

In the next section, we identify the subsample that can potentially be
brought out of poverty.

\section{Identifying the Problem}\label{identifying-the-problem}

The scope of the redistribution of household production responsibilities
in reducing time poverty depends on the living arrangements of
individuals. If the time-poor household consists only of one working-age
individual, no redistribution is possible. For example, a family of an
employed single mother and her young children is an important type of
family that is quite vulnerable to time poverty yet falls outside the
scope of redistribution. Households with two or more working-age persons
can vary in terms of the characteristics relevant to time poverty and
redistribution. Further, time-poor households in which all working-age
persons are time-poor represent a case where individual and collective
interests may not coincide. In these households, redistribution is
possible but ineffective in reducing the household's time deficit.
Hence, redistribution cannot change the household's status of time or
consumption poverty. A married (cohabitating) couple with young children
where both spouses encounter time deficits while managing their jobs and
household responsibilities fits this description. However,
redistribution can make time deficits less unequal between the time-poor
persons in this group of households, even potentially facilitating the
transition out of time poverty for some individuals. Greater gender
equality in the division of household responsibilities is desirable in
this group, too, because gender equality is intrinsically important,
irrespective of whether it affects the household's consumption poverty
or time poverty status.

We specify our problem statement to bring households out of poverty and
focus on only those households which can be brought of poverty. For this
purpose, we identify the households with atleast one time poor and one
time non-poor adult individual for us to redistribute effectively so
that the hh can fall out of poverty. This excludes: i) single individual
housheold (ii) hosueholds with irreducible time poverty (i.e housheolds
where total time surplus falls short of total tme deficit, thereby not
allowing for any effective redistribution). By the same logic we end up
identifying households where we can : (i) Reduce individual time poverty
{[}{]} (ii) Reduce share of hosueholds experienecing time poverty

In (\textbf{fig\_idsample?}) xxx

\begin{itemize}
\tightlist
\item
  The problem we need to identify the problem of time-poverty caused by
  redistribution (or lack thereof) of household production.
\item
  Identify either: how many Time poor individuals live in household with
  time non-poor adults.
\item
  Or identify the baseline of time poverty if there is full flexibility
  for time allocation. (Household Deficit consider both time deficits
  and surpluses)
\end{itemize}

This would give us a fist look at how much poverty could be alleviated
if household production was redistributed.

We could even look at Who are this individuals who are living in time
poverty, but that Do not need to. (describe the characteristics of these
individuals)

\begin{itemize}
\item
  Added value. This will help us identify those who cannot be helped by
  redistribution of household production. (even if their incidence
  changes
\item
  This raises the question. Do we want to analyze redistribution in
  household that are not time poor?
\end{itemize}

Next, we develop three redistribution scenarios wherein we redistribute
the share of household production responsibilities among the household's
working-age memebrs while all else remains. In other words, we examine
the effects of a new set of values of \(\alpha_{ij}\). Redistribution
will change the magnitude of the time balanace in Equation~\ref{eq-bal}
and, depending on the extent and direction of the change, may result in
a time-poor person becoming time-nonpoor or a time-nonpoor person
experiencing time poverty. We focus on the subsample of households for
which redistribution is possible and efficient. For example
redistributing househodd prod in a hh with atleast one time poor and one
time non poor individual along with the constraint that redistribution
can take place to adults belonging to 18-64 years. This allows for us to
not redistribute household production to children.

\section{Redistribution Scenarios}\label{redistribution-scenarios}

\begin{itemize}
\item
  Here we would describe the three redistribution scenarios we have
  developed. This would be ``realistic'' scenarios.
\item
  Describe the scenarios and the assumptions behind them.
\end{itemize}

Intrahousehold redistribution can potentially reduce time deficits. We
construct three redistrubution scenarios based on different guiding
principles. The extent of the reduction would depend on the principle
that we use in distributing household responsibilities among the
members. First, we use the simple egalitarianism principle that involves
an equal division of total household productiom time among all working
age members. Second, we

xxxx We also briefly outline the methods used for implementing the
principles in our data, with the detailed explanation of some aspects
provided in Appendix xx. Next, we provide an assessment of the different
principles in terms of how far they improve the position of women and
how much such improvements are congruent with the betterment of the
economic well-being of their families. In the subsequent section, we
compare and contrast the joint distribution of time and consumption
poverty among families and individuals that would result from each
principle.

\subsection{Distribution Rules for Household
Production}\label{distribution-rules-for-household-production}

Alternative values of \(\alpha_{ij}\) indicate how household production
requirements, net of the portion met by household members that are not
of working age or are physically unable to take on more work, are shared
between working-age persons in the household. Below we discuss the three
principles. \#\#\# 1. Equal Shares Scenario The procedure for the equal
shares scenario is relatively simple. Recall that the shares of those in
the redistribution simulation in this scenario are simply:

As before, denoting the number of working-age persons in household \(j\)
as \(I^j\), we can express the rule as:

\begin{equation}\phantomsection\label{eq-r1}{$\alpha_{ij}$^E= 1/$I^j$ *(1-$\alpha_{j}$^nw')
}\end{equation}

We need to count how many people are in the redistribution simulation in
each household and then assign them the appropriate fraction (1 for
households with one person in the simulation, 1⁄2 for households with
two people in the simulation, and so on) and apply that fraction to the
redistributable share of required household production time.

\subsubsection{2. Time Available Scenario (WIP
editing)}\label{time-available-scenario-wip-editing}

The time available scenario is based on equity such that the
redistrubted shares are based on the time that is available after
setting aside the time for personal maintenance requirements and income
generation. In other words, the household members should split up the
required household production time based on the time each one has
available, i.e based on an equity criteria. The time available
(\(Z_{ij}\))) is defined as the time left over after the minimum
personal maintenance and time spent on income generation (including
commuting time) have been subtracted from the total weekly hours. To
calculate the shares for each individual based on this principle, we
first calculate the time available for each individual, then add up the
total among the household for those individuals that have positive time
available. We then divide each individual's time available by the total
and apply that fraction to the redistributable share of household
production time. For those individuals that have negative time available
we set their shares to zero in this simulation.

\subsubsection{3. Opportunity costs}\label{opportunity-costs}

The third possibility is based on the idea of opportunity costs along
marginalist lines. The sharing rule depends on the relative actual
(potential) wage. For example, if there are only two working-age adults,
say husband and wife, and if the husband's wage is twice as much as the
wife, the wife's share would be two-thirds and the husband's share would
be one-third. Thus:

For the opportunity cost scenario, we imputed wages for all of those not
currently working for pay. In order to do this, we used a two-stage
Heckman selection model (Heckman 1979), also known as the Heckit
procedure, which we outline below. Once done, we used the imputed wages
of those that are not currently working for pay and the actual wages of
those that are to divide up the redistributable share of required
household production:

As the share of required household production needs to be inversely
proportional to the individual's share of the sum of wages, we subtract
their share of this sum from one. To ensure that the resulting shares
sum up to unity, we divide by the number of individuals in the
simulation minus one. We then apply this share to the redistributable
share of required household production as in previous steps. In order to
impute wages for those not currently employed for wages, we first impute
the likeliest industry and occupation for each individual using a
multinomial probit procedure. Industry and occupation are regressed on
age, age squared, sex, rural/urban status, education, and geographic
region on all those employed for wages. The likelihood for each industry
and occupation is then predicted for everyone, using the results of the
multinomial probit. Then each individual not currently working for wages
is assigned the industry and occupation corresponding to the largest
predicted likelihoods for that individual.

Describe our procedue : 2 stages

We simulate each of these principles of redistribution and recalculate
individual and household time and income poverty using the LIMTP
framework described above.

\section{Results}\label{results}

\begin{itemize}
\item
  Compare the time poverty changes on those identified earlier.
\item
  Moreover, we examine the changes by sex, employment status, presence
  of children in the household.
\item
  Perhaps Start with a global analysis (without Specific groups)
\item
  then Analyze the case for Specific groups
\end{itemize}

Perhaps provide more emphasis on Some of the groups (gender, the
employed, parents?)

\subsection{Gender Disparity in the Incidence of Time
Deficits}\label{gender-disparity-in-the-incidence-of-time-deficits}

Evidence indicates that negative time balance values (i.e., time
deficits) occur mostly among employed persons Add table by sex and
employment status for limtip and discuss results. Further, presence of
children would demand more caregiving hours thereby putting a pressure
on time available, particulalrly for employed couples or single memebr
hosuehodls. Add table by sex and children for limtip and discuss
results.

\#\#Gender and race Disparity in the Incidence of Time Deficits

\section{Policy implications}\label{policy-implications}

\section{Conclusion}\label{conclusion}



\end{document}
