% Options for packages loaded elsewhere
\PassOptionsToPackage{unicode}{hyperref}
\PassOptionsToPackage{hyphens}{url}
\PassOptionsToPackage{dvipsnames,svgnames,x11names}{xcolor}
%
\documentclass[
  11pt,
]{article}

\usepackage{amsmath,amssymb}
\usepackage{iftex}
\ifPDFTeX
  \usepackage[T1]{fontenc}
  \usepackage[utf8]{inputenc}
  \usepackage{textcomp} % provide euro and other symbols
\else % if luatex or xetex
  \usepackage{unicode-math}
  \defaultfontfeatures{Scale=MatchLowercase}
  \defaultfontfeatures[\rmfamily]{Ligatures=TeX,Scale=1}
\fi
\usepackage{lmodern}
\ifPDFTeX\else  
    % xetex/luatex font selection
\fi
% Use upquote if available, for straight quotes in verbatim environments
\IfFileExists{upquote.sty}{\usepackage{upquote}}{}
\IfFileExists{microtype.sty}{% use microtype if available
  \usepackage[]{microtype}
  \UseMicrotypeSet[protrusion]{basicmath} % disable protrusion for tt fonts
}{}
\makeatletter
\@ifundefined{KOMAClassName}{% if non-KOMA class
  \IfFileExists{parskip.sty}{%
    \usepackage{parskip}
  }{% else
    \setlength{\parindent}{0pt}
    \setlength{\parskip}{6pt plus 2pt minus 1pt}}
}{% if KOMA class
  \KOMAoptions{parskip=half}}
\makeatother
\usepackage{xcolor}
\usepackage[lmargin=1in,rmargin=1in,tmargin=1in,bmargin=1in]{geometry}
\setlength{\emergencystretch}{3em} % prevent overfull lines
\setcounter{secnumdepth}{3}
% Make \paragraph and \subparagraph free-standing
\makeatletter
\ifx\paragraph\undefined\else
  \let\oldparagraph\paragraph
  \renewcommand{\paragraph}{
    \@ifstar
      \xxxParagraphStar
      \xxxParagraphNoStar
  }
  \newcommand{\xxxParagraphStar}[1]{\oldparagraph*{#1}\mbox{}}
  \newcommand{\xxxParagraphNoStar}[1]{\oldparagraph{#1}\mbox{}}
\fi
\ifx\subparagraph\undefined\else
  \let\oldsubparagraph\subparagraph
  \renewcommand{\subparagraph}{
    \@ifstar
      \xxxSubParagraphStar
      \xxxSubParagraphNoStar
  }
  \newcommand{\xxxSubParagraphStar}[1]{\oldsubparagraph*{#1}\mbox{}}
  \newcommand{\xxxSubParagraphNoStar}[1]{\oldsubparagraph{#1}\mbox{}}
\fi
\makeatother


\providecommand{\tightlist}{%
  \setlength{\itemsep}{0pt}\setlength{\parskip}{0pt}}\usepackage{longtable,booktabs,array}
\usepackage{calc} % for calculating minipage widths
% Correct order of tables after \paragraph or \subparagraph
\usepackage{etoolbox}
\makeatletter
\patchcmd\longtable{\par}{\if@noskipsec\mbox{}\fi\par}{}{}
\makeatother
% Allow footnotes in longtable head/foot
\IfFileExists{footnotehyper.sty}{\usepackage{footnotehyper}}{\usepackage{footnote}}
\makesavenoteenv{longtable}
\usepackage{graphicx}
\makeatletter
\def\maxwidth{\ifdim\Gin@nat@width>\linewidth\linewidth\else\Gin@nat@width\fi}
\def\maxheight{\ifdim\Gin@nat@height>\textheight\textheight\else\Gin@nat@height\fi}
\makeatother
% Scale images if necessary, so that they will not overflow the page
% margins by default, and it is still possible to overwrite the defaults
% using explicit options in \includegraphics[width, height, ...]{}
\setkeys{Gin}{width=\maxwidth,height=\maxheight,keepaspectratio}
% Set default figure placement to htbp
\makeatletter
\def\fps@figure{htbp}
\makeatother

\makeatletter
\@ifpackageloaded{caption}{}{\usepackage{caption}}
\AtBeginDocument{%
\ifdefined\contentsname
  \renewcommand*\contentsname{Table of contents}
\else
  \newcommand\contentsname{Table of contents}
\fi
\ifdefined\listfigurename
  \renewcommand*\listfigurename{List of Figures}
\else
  \newcommand\listfigurename{List of Figures}
\fi
\ifdefined\listtablename
  \renewcommand*\listtablename{List of Tables}
\else
  \newcommand\listtablename{List of Tables}
\fi
\ifdefined\figurename
  \renewcommand*\figurename{Figure}
\else
  \newcommand\figurename{Figure}
\fi
\ifdefined\tablename
  \renewcommand*\tablename{Table}
\else
  \newcommand\tablename{Table}
\fi
}
\@ifpackageloaded{float}{}{\usepackage{float}}
\floatstyle{ruled}
\@ifundefined{c@chapter}{\newfloat{codelisting}{h}{lop}}{\newfloat{codelisting}{h}{lop}[chapter]}
\floatname{codelisting}{Listing}
\newcommand*\listoflistings{\listof{codelisting}{List of Listings}}
\makeatother
\makeatletter
\makeatother
\makeatletter
\@ifpackageloaded{caption}{}{\usepackage{caption}}
\@ifpackageloaded{subcaption}{}{\usepackage{subcaption}}
\makeatother

\ifLuaTeX
  \usepackage{selnolig}  % disable illegal ligatures
\fi
\usepackage{bookmark}

\IfFileExists{xurl.sty}{\usepackage{xurl}}{} % add URL line breaks if available
\urlstyle{same} % disable monospaced font for URLs
\hypersetup{
  pdftitle={Redistribution and Time Poverty: Balancing Responsibilities in Couple Households},
  pdfauthor={Fernando Rios-Avila; Aashima Sinha},
  pdfkeywords={Time Poverty, Income Poverty, Redistribution, household
production, care work, gender equality, LIMTIP},
  colorlinks=true,
  linkcolor={blue},
  filecolor={Maroon},
  citecolor={Blue},
  urlcolor={Blue},
  pdfcreator={LaTeX via pandoc}}



\usepackage{datetime}
\usepackage{booktabs}
\usepackage{chngcntr}
\usepackage{apptools}
\usepackage{lipsum}
\usepackage{booktabs}
\usepackage{multirow}
\usepackage{makecell}
\AtAppendix{\counterwithin{table}{section}}
\AtAppendix{\counterwithin{figure}{section}}

\title{Redistribution and Time Poverty: Balancing Responsibilities in
Couple Households}
\author{
Fernando Rios-Avila\\
Levy Economics Institute\\
\\
\and 
Aashima Sinha\\
Levy Economics Institute\\
\\
}
\date{2024-10-17}
\begin{document}


\def\spacingset#1{\renewcommand{\baselinestretch}%
{#1}\small\normalsize} \spacingset{1}

%Ipsum lorem

\maketitle
\begin{abstract}
To be rewritten
\end{abstract}
 
\vspace{.2in}

\textbf{\textit{Keyword: }}Time Poverty, Income Poverty, Redistribution,
household production, care work, gender equality, LIMTIP


\thispagestyle{empty}
\clearpage\pagenumbering{arabic}
\newpage
\spacingset{1.2} % DON'T change the spacing!

\section{Introduction}\label{introduction}

RRR and time poverty

What is LIMTIP? What is Time Poverty?

Explain LIMTIP (Levy Institute Measure of Time and Income Poverty),
focusing on how time poverty affects households with couples. Introduce
the main question: ``To what extent can the redistribution of
responsibilities within households reduce time poverty?''

\section{Where we are, where we are going: Redistribution
Scenarios}\label{where-we-are-where-we-are-going-redistribution-scenarios}

Explain the three redistribution scenarios: equality, equity, and
opportunity cost principles.

Also explain Briefly how LIMTIP is used to measure time poverty.

Provide Baseline Statistics for Time poverty in the US. Include basic
statistics on time poverty in the US.

Also explain the data used for the analysis. Also ID restrictions for
the analysis.

\begin{table}

\caption{\label{tbl-stat}Summary Statisitcs Population}

\centering{

\resizebox{\textwidth}{!}{%

\begin{tabular}{l*{7}{c}}
\hline\hline
            & All Mem. TP&At Least 1 Mem. NTP&Hhld can exit TP&Has Y. Children&Oth Mem Present&  H. Working&  W. Working\\
\hline
All         &         5.2&        18.9&        75.9&        55.8&        25.3&        97.2&        91.3\\
Has Y. Children&         7.3&        27.2&        65.5&       100.0&        17.9&        97.3&        89.3\\
No Y. Children&         2.6&         8.3&        89.1&         0.0&        34.6&        97.2&        93.9\\
Other H Member&         0.1&         2.6&        97.3&        39.5&       100.0&        96.4&        88.9\\
No Other Member&         7.0&        24.4&        68.7&        61.3&         0.0&        97.5&        92.1\\
Wife Works  &         5.7&        20.4&        73.9&        54.5&        24.7&        97.1&       100.0\\
Wife Not Working&         0.4&         2.8&        96.9&        69.0&        32.2&        98.3&         0.0\\
\hline\hline
\end{tabular}


}

}

\end{table}%

\section{Impact of Redistribution on Time Poverty: Time Povery and time
deficits}\label{impact-of-redistribution-on-time-poverty-time-povery-and-time-deficits}

Time Poverty and transition rates

\begin{table}

\caption{\label{tbl-tb2}Time Poverty and Transition Rates}

\centering{

\resizebox{\textwidth}{!}{%

\begin{tabular}{l*{8}{c}}
\hline\hline
          & \multicolumn{4}{c}{Men} & \multicolumn{4}{c}{Wife} \\ 
\cline{2-5} \cline{6-9}
            &    Baseline&  Scenario 1&  Scenario 2&  Scenario 3&    Baseline&  Scenario 1&  Scenario 2&  Scenario 3\\
\hline
All         &        43.8&        38.7&        23.8&        26.2&        61.0&        18.6&        22.2&        31.5\\
BL: Time NP &         0.0&        23.2&        19.3&        13.7&         0.0&         6.6&        15.6&        16.3\\
BL: Time P  &       100.0&        58.5&        29.6&        42.1&       100.0&        26.3&        26.4&        41.2\\
\midrule

Household Type      & \multicolumn{8}{c}{} \\ 
All Mem. TP         &       100.0&        95.8&        98.3&        82.4&       100.0&        68.7&        97.4&        81.8\\
At Least 1 Mem. NTP &        40.1&        82.4&        82.8&        55.8&        61.0&        53.5&        80.6&        65.1\\
Hhld can exit TP    &        40.9&        23.9&         4.1&        14.9&        58.4&         6.5&         2.4&        19.7\\
\hline\hline
\end{tabular}


}

}

\end{table}%

Perhaps add a table with all changes in time poverty for each scenario
and sub groups. Or do 3 , one for each scenario.

\begin{table}

\caption{\label{tbl-tb3}Time Poverty by Household Structure}

\centering{

\resizebox{\textwidth}{!}{%

\begin{tabular}{l*{8}{c}}
\hline\hline
          & \multicolumn{4}{c}{Men} & \multicolumn{4}{c}{Wife} \\  \cline{2-5} \cline{6-9}
            &    Baseline&  Scenario 1&  Scenario 2&  Scenario 3&    Baseline&  Scenario 1&  Scenario 2&  Scenario 3\\
Yng Children Presence & \multicolumn{8}{c}{} \\ 
\ \ No Children&        43.1&        24.1&        13.0&        17.0&        59.0&        11.4&        11.4&        19.6\\
\ \ With Children&        44.4&        50.2&        32.5&        33.4&        62.6&        24.4&        30.7&        41.0\\
\midrule

Other Members in HH    & \multicolumn{8}{c}{} \\ 
\ \ No      &        44.4&        43.8&        29.4&        31.8&        62.0&        21.9&        27.5&        39.1\\
\ \ Yes     &        41.9&        23.6&         7.5&         9.6&        58.1&         9.0&         6.3&         9.3\\
\hline\hline
\end{tabular}


}

}

\end{table}%

Adjusted Income Poverty

\begin{table}

\caption{\label{tbl-tb4}Time Poverty by Income}

\centering{

\resizebox{\textwidth}{!}{%

\begin{tabular}{l*{8}{c}}
\hline\hline
          & \multicolumn{4}{c}{Men} & \multicolumn{4}{c}{Wife} \\  \cline{2-5} \cline{6-9}
            &    Baseline&  Scenario 1&  Scenario 2&  Scenario 3&    Baseline&  Scenario 1&  Scenario 2&  Scenario 3\\
Income/Pline   & \multicolumn{8}{c}{} \\ 
\ \ < PLine &        44.7&        35.3&        12.1&        24.3&        55.9&        10.8&        11.1&        20.5\\
\ \ 1-2 x Pline&        42.0&        39.4&        19.8&        25.4&        59.9&        16.8&        17.7&        28.6\\
\ \ 2-4 x Pline&        42.9&        38.2&        24.7&        26.1&        61.7&        18.7&        22.9&        31.1\\
\ \ >4 x Pline&        45.9&        39.1&        26.1&        26.8&        61.3&        20.4&        24.7&        34.7\\
\midrule

Wife Work Status   & \multicolumn{8}{c}{} \\ 
\ \ Not Working&        83.3&        65.9&         8.6&        48.9&         9.9&         0.0&         2.5&         0.1\\
\ \ Working &        40.0&        36.1&        25.3&        24.0&        65.9&        20.4&        24.0&        34.5\\
\hline\hline
\end{tabular}


}

}

\end{table}%

Statistics on the hidden poor

\begin{table}

\caption{\label{tbl-tb5}Hidden Poor by Characteristics}

\centering{

\resizebox{\textwidth}{!}{%

\begin{tabular}{l*{4}{c}}
\hline\hline
            &    Baseline&  Scenario 1&  Scenario 2&  Scenario 3\\
\hline
All         &         4.7&         1.8&         0.7&         1.8\\
\midrule

Household Type      & \multicolumn{4}{c}{} \\ 
\ \ All Mem. TP         &         5.3&         5.9&         5.7&         5.8\\
\ \ At Least 1 Mem. NTP &         6.9&         3.2&         1.8&         3.4\\
\ \ Hhld can exit TP    &         4.1&         1.2&         0.1&         1.1\\
\midrule

Yng Children Presence & \multicolumn{4}{c}{} \\ 
\ \ No Children &         2.5&         0.7&         0.2&         0.7\\
\ \ With Children &         6.6&         2.7&         1.1&         2.6\\
\midrule

Other Members in HH      & \multicolumn{4}{c}{} \\ 
\ \ No      &         4.5&         2.1&         0.9&         2.1\\
\ \ Yes     &         5.5&         1.1&         0.2&         0.9\\
\midrule

Income/Pline     & \multicolumn{4}{c}{} \\ 
\ \ 1-2 x Pline&        22.9&         8.6&         3.3&         8.3\\
\ \ 2-4 x Pline&         0.3&         0.2&         0.1&         0.2\\
\midrule

Wife Work Status  & \multicolumn{4}{c}{} \\ 
\ \ Not Working        &         7.8&         7.6&         0.5&         5.1\\
\ \ Working &         4.5&         1.3&         0.7&         1.5\\
\hline\hline
\end{tabular}


}

}

\end{table}%

\section{Policy Implications: Opportunities and
Challenges}\label{policy-implications-opportunities-and-challenges}

Redistribution can reduce time poverty, but only so much

Which is more effective? Are the results consistent with expectations?

What about based on household characteristics?

\section{Conclusion/recommendations}\label{conclusionrecommendations}




\end{document}
