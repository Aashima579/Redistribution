% Options for packages loaded elsewhere
\PassOptionsToPackage{unicode}{hyperref}
\PassOptionsToPackage{hyphens}{url}
\PassOptionsToPackage{dvipsnames,svgnames,x11names}{xcolor}
%
\documentclass[
  11pt,
]{article}

\usepackage{amsmath,amssymb}
\usepackage{iftex}
\ifPDFTeX
  \usepackage[T1]{fontenc}
  \usepackage[utf8]{inputenc}
  \usepackage{textcomp} % provide euro and other symbols
\else % if luatex or xetex
  \usepackage{unicode-math}
  \defaultfontfeatures{Scale=MatchLowercase}
  \defaultfontfeatures[\rmfamily]{Ligatures=TeX,Scale=1}
\fi
\usepackage{lmodern}
\ifPDFTeX\else  
    % xetex/luatex font selection
\fi
% Use upquote if available, for straight quotes in verbatim environments
\IfFileExists{upquote.sty}{\usepackage{upquote}}{}
\IfFileExists{microtype.sty}{% use microtype if available
  \usepackage[]{microtype}
  \UseMicrotypeSet[protrusion]{basicmath} % disable protrusion for tt fonts
}{}
\makeatletter
\@ifundefined{KOMAClassName}{% if non-KOMA class
  \IfFileExists{parskip.sty}{%
    \usepackage{parskip}
  }{% else
    \setlength{\parindent}{0pt}
    \setlength{\parskip}{6pt plus 2pt minus 1pt}}
}{% if KOMA class
  \KOMAoptions{parskip=half}}
\makeatother
\usepackage{xcolor}
\usepackage[lmargin=1in,rmargin=1in,tmargin=1in,bmargin=1in]{geometry}
\setlength{\emergencystretch}{3em} % prevent overfull lines
\setcounter{secnumdepth}{3}
% Make \paragraph and \subparagraph free-standing
\ifx\paragraph\undefined\else
  \let\oldparagraph\paragraph
  \renewcommand{\paragraph}[1]{\oldparagraph{#1}\mbox{}}
\fi
\ifx\subparagraph\undefined\else
  \let\oldsubparagraph\subparagraph
  \renewcommand{\subparagraph}[1]{\oldsubparagraph{#1}\mbox{}}
\fi


\providecommand{\tightlist}{%
  \setlength{\itemsep}{0pt}\setlength{\parskip}{0pt}}\usepackage{longtable,booktabs,array}
\usepackage{calc} % for calculating minipage widths
% Correct order of tables after \paragraph or \subparagraph
\usepackage{etoolbox}
\makeatletter
\patchcmd\longtable{\par}{\if@noskipsec\mbox{}\fi\par}{}{}
\makeatother
% Allow footnotes in longtable head/foot
\IfFileExists{footnotehyper.sty}{\usepackage{footnotehyper}}{\usepackage{footnote}}
\makesavenoteenv{longtable}
\usepackage{graphicx}
\makeatletter
\def\maxwidth{\ifdim\Gin@nat@width>\linewidth\linewidth\else\Gin@nat@width\fi}
\def\maxheight{\ifdim\Gin@nat@height>\textheight\textheight\else\Gin@nat@height\fi}
\makeatother
% Scale images if necessary, so that they will not overflow the page
% margins by default, and it is still possible to overwrite the defaults
% using explicit options in \includegraphics[width, height, ...]{}
\setkeys{Gin}{width=\maxwidth,height=\maxheight,keepaspectratio}
% Set default figure placement to htbp
\makeatletter
\def\fps@figure{htbp}
\makeatother
% definitions for citeproc citations
\NewDocumentCommand\citeproctext{}{}
\NewDocumentCommand\citeproc{mm}{%
  \begingroup\def\citeproctext{#2}\cite{#1}\endgroup}
\makeatletter
 % allow citations to break across lines
 \let\@cite@ofmt\@firstofone
 % avoid brackets around text for \cite:
 \def\@biblabel#1{}
 \def\@cite#1#2{{#1\if@tempswa , #2\fi}}
\makeatother
\newlength{\cslhangindent}
\setlength{\cslhangindent}{1.5em}
\newlength{\csllabelwidth}
\setlength{\csllabelwidth}{3em}
\newenvironment{CSLReferences}[2] % #1 hanging-indent, #2 entry-spacing
 {\begin{list}{}{%
  \setlength{\itemindent}{0pt}
  \setlength{\leftmargin}{0pt}
  \setlength{\parsep}{0pt}
  % turn on hanging indent if param 1 is 1
  \ifodd #1
   \setlength{\leftmargin}{\cslhangindent}
   \setlength{\itemindent}{-1\cslhangindent}
  \fi
  % set entry spacing
  \setlength{\itemsep}{#2\baselineskip}}}
 {\end{list}}
\usepackage{calc}
\newcommand{\CSLBlock}[1]{\hfill\break\parbox[t]{\linewidth}{\strut\ignorespaces#1\strut}}
\newcommand{\CSLLeftMargin}[1]{\parbox[t]{\csllabelwidth}{\strut#1\strut}}
\newcommand{\CSLRightInline}[1]{\parbox[t]{\linewidth - \csllabelwidth}{\strut#1\strut}}
\newcommand{\CSLIndent}[1]{\hspace{\cslhangindent}#1}

\makeatletter
\@ifpackageloaded{caption}{}{\usepackage{caption}}
\AtBeginDocument{%
\ifdefined\contentsname
  \renewcommand*\contentsname{Table of contents}
\else
  \newcommand\contentsname{Table of contents}
\fi
\ifdefined\listfigurename
  \renewcommand*\listfigurename{List of Figures}
\else
  \newcommand\listfigurename{List of Figures}
\fi
\ifdefined\listtablename
  \renewcommand*\listtablename{List of Tables}
\else
  \newcommand\listtablename{List of Tables}
\fi
\ifdefined\figurename
  \renewcommand*\figurename{Figure}
\else
  \newcommand\figurename{Figure}
\fi
\ifdefined\tablename
  \renewcommand*\tablename{Table}
\else
  \newcommand\tablename{Table}
\fi
}
\@ifpackageloaded{float}{}{\usepackage{float}}
\floatstyle{ruled}
\@ifundefined{c@chapter}{\newfloat{codelisting}{h}{lop}}{\newfloat{codelisting}{h}{lop}[chapter]}
\floatname{codelisting}{Listing}
\newcommand*\listoflistings{\listof{codelisting}{List of Listings}}
\makeatother
\makeatletter
\makeatother
\makeatletter
\@ifpackageloaded{caption}{}{\usepackage{caption}}
\@ifpackageloaded{subcaption}{}{\usepackage{subcaption}}
\makeatother
\ifLuaTeX
  \usepackage{selnolig}  % disable illegal ligatures
\fi
\usepackage{bookmark}

\IfFileExists{xurl.sty}{\usepackage{xurl}}{} % add URL line breaks if available
\urlstyle{same} % disable monospaced font for URLs
\hypersetup{
  pdftitle={Redistribution and Time Poverty: Balancing Responsibilities in Couple Households},
  pdfauthor={Fernando Rios-Avila; Aashima Sinha},
  pdfkeywords={Time Poverty, Income Poverty, Redistribution, household
production, care work, gender equality, LIMTIP},
  colorlinks=true,
  linkcolor={blue},
  filecolor={Maroon},
  citecolor={Blue},
  urlcolor={Blue},
  pdfcreator={LaTeX via pandoc}}


\usepackage{datetime}
\usepackage{booktabs}
\usepackage{chngcntr}
\usepackage{apptools}
\usepackage{lipsum}
\usepackage{booktabs}
\usepackage{multirow}
\usepackage{makecell}
\AtAppendix{\counterwithin{table}{section}}
\AtAppendix{\counterwithin{figure}{section}}

\title{Redistribution and Time Poverty: Balancing Responsibilities in
Couple Households}
\author{
Fernando Rios-Avila\\
Levy Economics Institute\\
\\
\and 
Aashima Sinha\\
Levy Economics Institute\\
\\
}
\date{2024-10-22}
\begin{document}


\def\spacingset#1{\renewcommand{\baselinestretch}%
{#1}\small\normalsize} \spacingset{1}

%Ipsum lorem

\maketitle
\begin{abstract}
This policy brief examines the potential of redistributing household
production responsibilities to alleviate time poverty for married
couples in the United States. Using the Levy Institute Measure of Time
and Income Poverty (LIMTIP), the brief explores three redistribution
scenarios based on equality, equity, and opportunity cost principles. We
focus on what happens to married men and women by assessing their time
poverty, time deficits, transition rates out of poverty among other
outcomes. Findings show that redistribution can be an effective tool to
reduce time poverty, particularly in households with children and where
other household members are present. The equity-based approach emerges
as most effective in reducing poverty rates overall. Redistribution is
found to be less impactful in households where all members are already
time-poor suggesting intra-household redistribution may not be
effective, whereas for household types where there is a mmix of time
poor and non-poor such that time surplus outweighs time defeicits,
redistribution can serve to bring married couples out of poverty. The
brief highlights how redistribution can promote more equitable sharing
of responsibilities between married men and women and potentially lift
entire households out of poverty. However, effects vary across the
structure of the household and different scenarios, suggesting a
one-size-fits-all approach may not be optimal.
\end{abstract}
 
\vspace{.2in}

\textbf{\textit{Keyword: }}Time Poverty, Income Poverty, Redistribution,
household production, care work, gender equality, LIMTIP


\thispagestyle{empty}
\clearpage\pagenumbering{arabic}
\newpage
\spacingset{1.2} % DON'T change the spacing!
\section{Introduction}\label{introduction}

Explain LIMTIP (Levy Institute Measure of Time and Income Poverty),
focusing on how time poverty affects households with couples. Introduce
the main question: ``To what extent can the redistribution of
responsibilities within households reduce time poverty?''

Redistribution of household production, which includes unpaid caregiving
and domestic chores, has been identified as an important tool to achieve
gender equality. The United Nations Sustainable Development Goal 5,
Target 5.4, has incorporated the recognition, reduction, and
redistribution of unpaid work staretgy, popularly known as the 3R
startegy, which is a testament to decades of activism and advocacy
emphasizing that gender inequality on this front cannot be justified as
a ``private family matter'' rather is a matter of public policy.
Redistribution can take place from households to the public and/or
private spheres, as well as among household members. While all household
members may share household work, evidence shows that it is
disproportionately undertaken by girls and women globally (Addati et
al., 2018).

Redistribution of household production responsibilities from women to
men is important intrinsically for human rights and fairness concerns;
it is also instrumental in achieving gender equality in labor market
outcomes (Bruyn-Hundt, 1996; Elso, 2017; Esquivel, 2016). Studies have
demonstrated that gender gaps in the workforce and the unequal sharing
of household responsibilities can severely impede economic growth and
development (Berik et al., 2009; Duflo, 2012; Elson, 2009). Yet, public
policies and collective actions have been less than adequate, especially
in poorer countries with constrained fiscal capacity, widespread absence
of formal wage labor, and weak welfare states. Moreover, in patriarchal
contexts, cultural barriers restrict redistribution of household
production, particularly unpaid care work from women to men and to the
public and private spheres. While in some developed countries such as
Norway and Sweden, public policies have been able to promote
gender-equitable sharing of household production, such as paid paternity
leaves in addition to paid maternity leaves, they have attained limited
attention and success in other countries.

The U.S. is not an exception. Issues related to lack of public
provisioning of care infrastructure and services, widespread existence
of childcare deserts, and lack of paid parental leave laws, among
others, have drawn attention. In 2021, the value of unpaid household
work in the U.S. amounted to \$600 billion, constituting approximately
2.6\% of the GDP (Reinhard et al., 2023). Moreover, like most other
countries, we observe gender disparity in sharing of household work such
that women disproportionately shoulder the burden. According to the 2018
American Time Use Survey, among adults aged 15 and older, women on
average spent 5.7 hours per day on unpaid household and care work,
compared with 3.6 hours for men. In other words, women spent 37 percent
more time on unpaid household and care work than men (Hess et al.,
2020). The disparity is more significant between married men and married
women xxxx(TBD) and also in households where children are present.
Additionally, the U.S. falls behind many OECD countries in effective
childcare policies, spending only 0.4\% of GDP on early childhood
education and care (ECEC), compared to the OECD average of 0.8\% (OECD,
2020). Notably, the U.S. lacks federal laws granting paid parental
leave, setting it apart from other OECD nations. Around 51\% of the U.S.
population resides in childcare deserts, defined as census tracts with
more than 50 children under the age of 5 and either no childcare
providers or significantly limited options, resulting in a severe
shortage of licensed child care slots (Malik et al., 2018).

\subsection{What does this means for time
poverty?}\label{what-does-this-means-for-time-poverty}

The lack of public provisioning of care infrastructure and services, and
the disproportionate burden of household production on women, has
implications for time poverty, both at the individual and the
household/family level. Individual time poverty refers to the lack of
time available for individuals to engage in activities that are
essential for taking care of the household, its members, self-care, and
paid work. At the household level, even if a single individual struggles
to meet their responsibilities, the whole family is considered to be
living under time poverty. In this framework, as pointed out in
(\textbf{policybrief\_USLIMTIP?}), it is not uncommon to see households
with a mixture of time availability (i.e deficts and surpluses) among
its members. In fact, just over 20\% of the working-age population are
not time-poor but live in a household where at least one person lives
under time poverty. In spite of the growing recognition of the
importance of time constraints and the responsibility of household
production, the issue of time poverty has received limited attention in
the U.S., partially due to data availability constraints.

The question remains, can redistribution of housheold responsibilities
reduce time poverty?

Over the last decades, the Levy Economic Institute has been at the
forefront of recognizing the importance of time for understanding income
and poverty dynamics (Zacharias, 2011). As part of this work, they
developed a new measure of poverty that incorporates the dimension of
time into traditional poverty measures: The Levy Institute Measure of
Time and Income Poverty (LIMTIP for short). This measure uses synthetic
data in order to incorporate the value of time, or more specifically the
amount of resources required to outsource the responsibilities that
cannot be covered by the household members, into traditional measures of
poverty thresholds. By incorporating this dimension, the LIMTIP not only
provides a more comprehensive understanding of poverty but also allows
for the identification of the hidden poor, i.e., individuals whose
families do not have enough monetary resources to accommodate for the
time deficits they face (Antonopoulos et al., 2017; Masterson, 2012;
Zacharias et al., 2012, 2014, 2018, 2021).

While most of the earlier work on LIMTIP has focused on the analysis of
time poverty in developing countries (Masterson, 2012; Masterson et al.,
2022; Zacharias et al., 2018), recent work has extended the measure to
the U.S. (Zacharias et al., 2024;
\textbf{policybrief\_USLIMTIP?}).\footnote{This is in addition to the
  work done for the Levy Institute Measure of Economic Well-Being
  (LIMEW).} Similar to earlier work, one of the findings of
(\textbf{policybrief\_USLIMTIP?}) is that a large share of the
population experiences some level of time poverty, which translates into
a significant share of households who are \textbf{\emph{hidden poor}},
thus not captured by the official income poverty measure. In this policy
brief, we suggest that a significant share of time-poor individuals and
households could potentially exit time poverty if household production
responsibilities were to be redistributed among its members (similar to
Zacharias et al. (2021)).

Following (\textbf{policybrief\_USLIMTIP?}), this policy brief explores
the potential impacts of redistribution further. Using the new estimates
for LIMTIP for the U.S., we provide insights into how redistributing
household production can reduce the incidence of poverty not only for
individuals but also of the households they live in. Specifically, given
the marked responsibilities gap between men and women, we focus on
analyzing the benefits of redistribution among married couples. To do
this, we consider three redistribution scenarios based on equality,
equity, and opportunity cost principles and assess the impact of
redistribution on time poverty of working-age (18-64 years) household
members who are part of a heterosexual couple. Further, we present the
impacts for differnt household types, household structures (presence of
young childrena nd other members), poverty groups and employment status.

In the next section, we start by briefly describing the LIMTIP measure
and our estimates for the US. We then move on to identifying the
different types of households experiencing time poverty, the
redistribution scenarios, followed by results and policy implications.

\section{LIMTIP: A New Measure of Time Poverty for the United
States}\label{limtip-a-new-measure-of-time-poverty-for-the-united-states}

Poverty is a multidimensional concept that goes beyond the simple notion
of lack of income. In addition to income, poverty can be understood as a
lack of access to resources, including time. The LIMTIP is a metric
that, in addition to income poverty, incorporates aspects of time
poverty that better capture the control households have over their
resources. In this framework, time poverty refers to a scenario wherein
people may not have any time left after engaging in activities that are
essential for taking care of the household, its members, self-care, and
paid work. At the household level, we consider an even more restrictive
definition. Under the assumption that individuals with time surpluses
are unable or unwilling to help those with time deficits, we consider a
household to be time-poor if at least one member is time-poor.

As described in (\textbf{policybrief\_USLIMTIP?}) and
(\textbf{wp\_qmatch?}), the LIMTIP is built using a synthetic dataset
that combines information from the American Time Use Survey (ATUS) and
the Annual Social and Economic Supplements (ASEC) of the Current
Population Survey (CPS). For the identification of time poverty, using
weekly hours as the unit of analysis (168 hrs per week), we identify the
amount of time individuals would have left (\(X_{ij}\)) after engaging
in required activities for taking care of their share of
responsibilities (\(\alpha_{ij}\)) in household production (\(R_j\)),
personal maintenance (\(M\)), and paid work (commuting \(T_{ij}\) and
time spent at work \(L_{ij}\)). This is expressed in the following
equation (see Equation~\ref{eq-bal}):

\begin{equation}\phantomsection\label{eq-bal}{X_{ij} = 168 - M - \alpha_{ij}R_j-D_{ij}(L_{ij}+T_{ij})
}\end{equation}

The minimum time required for each of the components in
Equation~\ref{eq-bal} are estimated using a mixture of assumptions, the
synthetic dataset, and the ATUS dataset (see (\textbf{wp\_qmatch?}) for
details). An individual is classified as time poor if they have a a
negative time balance based on equation Equation~\ref{eq-bal}.

At the household level, however, we assume that individuals with time
surpluses are unable or unwilling to share and redistribute some of the
responsibilities of those with time deficits. In this framework, a
household is considered to be time-poor as long as there is atleast one
person with a time deficit living in the household.\footnote{To identify
  time poverty status, we only consider the time deficits of household
  members age 18 or older.} This is expressed in the following equation
(see Equation~\ref{eq-hbal}):

\begin{equation}\phantomsection\label{eq-hbal}{X_{j} = \sum_{i=1}^{I_j} \min(X_{ij},0)
}\end{equation}

Once household time deficits \(X_{j}\) are identified, we can adjust the
official income poverty thresholds to account for the monetized value of
the time deficits. For the U.S. case, we use a three-year average hourly
wage for the industry private households obtained from Merged Outgoing
Rotation Groups (MORG) to value the household time deficit. This value
represents the amount of income that may be required to outsource some
of the time responsibilities and eliminate time poverty. The adjusted
poverty line is then calculated as:

\begin{equation}\phantomsection\label{eq-limtip}{Z_{j}^{adj} = Z_{j} + 52*P* |X_{j}|
}\end{equation}

where \(P\) is the price we use to give a monetary value to the time
deficits the household \({j}\) faces, \(Z_{j}\) is the official poverty
line (SPM Poverty line), and \(Z_{j}^{adj}\) is the adjusted poverty
line. Intuitively, households that are not time-poor will not change
status compared to the official poverty estimates. However, households
that are time-poor could have their poverty status change if, after
considering the adjusted poverty line, they fall below it. This group of
households is considered to be the hidden poor.

\section{Where we are, where we are going: Redistribution
Scenarios}\label{where-we-are-where-we-are-going-redistribution-scenarios}

Explain the three redistribution scenarios: equality, equity, and
opportunity cost principles.

Also explain Briefly how LIMTIP is used to measure time poverty.

Provide Baseline Statistics for Time poverty in the US. Include basic
statistics on time poverty in the US.

Also explain the data used for the analysis. Also ID restrictions for
the analysis.

In the U.S, during 2015-19 on average 36.4\% households are time poor,
i.e atleast one member in these hh experiences time poverty. Within
these housheolds, 44\% married men and 61\% married women experience
time poverty. The shares of required hours of household production
reflects gender disparity in sharing of hh work and it has stayed more
or less stagnant during this period (add a figure 2015-19 srhp by sex).
In Figures 1 and 2, we present the shares of housheold production and
time defeicits experienced by married men and women. Further, in Table
xx we present the four-way classification of limtip estimates for
individuals. We find that xx \% women experience double poverty: income
poor and time poor. Given these time poverty outcomes, we investigate
the impact of redistribution of household production work among all
eligible members in time poor housheolds (18-64 and abled). Our focus is
on married couples and we examine if redistribution between them and to
other hh members can reduce time poverty. Moroever, 5.6\% of households
are hidden poor, i.e those who are not classified poor based on official
poverty estimates but are poor based on LIMTIP estimation. We finally
explore the potential of redistribution in lifting households out of
poverty and reducing the share of hidden poor.

\section{Household classification}\label{household-classification}

Households can vary in terms of the presence of time poor an dtime
non-poor members and in terms of the total time deficit and surplus.
Members with time surpluses could take on more household
responsibilities, reducing the burden of those with time deficits, and
potentially lifting the household out of poverty. Even if the household
remains time-poor, redistribution could still make the time deficits
more equal among the household members, particulalry balancing the share
between men and women.

We fisrt identify the households where redistribution is possible.
Specifically, we exclude from the analysis households that are not
time-poor, because the goal of the current analysis is to evaluate the
potential of redistribution to lift individuals and households out of
time poverty while reducing the gender time-poverty gaps. While non-time
poor households could potentially benefit from redistribution, reducing
the gaps of time surpluses among household members and/or resulting in
more gender-equitable sharing of household work, this is beyond the
current scope of this policy brief.

While we allow for redistribution to happen across all working-age
(18-64 years) and non-disabled household members (we call this group the
elgible sample), our main focus is on analyzing the impact of
redistribution between men and women. Thus, we will concentrate on
exploring the impact of redistribution on heterosexual couples, where
both partners are working-age, non-disabled individuals.

Under these considerations, we use an adaptation of the household
classification proposed in Zacharias et al. (2021), with two main
differences. First, we do not differentiate cases where, as described in
Zacharias et al. (2021), time poverty is driven by employment. Second,
for the household classification, we exclude the disabled when counting
the number of members for whom redistribution is possible within a
household. Thus, a ``single'' household refers to a household (SPM unit)
where there is only 1 working-age, non-disabled household member. Under
these considerations, Table~\ref{tbl-class} provides the household
classification used for the analysis.

\begin{longtable}[]{@{}
  >{\raggedright\arraybackslash}p{(\columnwidth - 4\tabcolsep) * \real{0.2273}}
  >{\raggedright\arraybackslash}p{(\columnwidth - 4\tabcolsep) * \real{0.1364}}
  >{\raggedright\arraybackslash}p{(\columnwidth - 4\tabcolsep) * \real{0.6364}}@{}}
\caption{Household Classification for Redistribution
Analysis}\label{tbl-class}\tabularnewline
\toprule\noalign{}
\begin{minipage}[b]{\linewidth}\raggedright
Household Type
\end{minipage} & \begin{minipage}[b]{\linewidth}\raggedright
Share
\end{minipage} & \begin{minipage}[b]{\linewidth}\raggedright
Description
\end{minipage} \\
\midrule\noalign{}
\endfirsthead
\toprule\noalign{}
\begin{minipage}[b]{\linewidth}\raggedright
Household Type
\end{minipage} & \begin{minipage}[b]{\linewidth}\raggedright
Share
\end{minipage} & \begin{minipage}[b]{\linewidth}\raggedright
Description
\end{minipage} \\
\midrule\noalign{}
\endhead
\bottomrule\noalign{}
\endlastfoot
Non-Time Poor & 61.3\% & Individuals living in households where no one
is time-poor. \\
Single Person Elig & 5.1\% & The household has only one working-age,
non-disabled individual. Redistribution is not possible. \\
HH Type I & 1.4\% & All members in the household are time-poor.
Household poverty cannot be eliminated, but individual time poverty
could change. \\
HH Type II & 5.2\% & There is at least one non-time poor individual
living in the household. Household poverty cannot be eliminated because
total household time surplus is less than total household time deficit,
however individual time poverty can be reduced. \\
HH Type III & 27.0\% & There is at least one non-time poor individual
living in the household. Household poverty can be eliminated because
total household time surplus is greater than total housheold time
deficit. \\
\end{longtable}

Across all these households, we will concentrate only on Households Type
I, II and III, since redistribution could modify the time poverty status
of individuals or households in these three categories.

In Table~\ref{tbl-stat} we present an overview of the sample
distribution, i.e share of individuals by hosuehold types,
characteristics and employment status of husband and wife (check with
FRA)

\begin{table}

\caption{\label{tbl-stat}Summary Statisitcs Population}

\centering{

\resizebox{\textwidth}{!}{%
\begin{tabular}{l*{7}{c}}
\hline\hline
            & All Mem. TP&At Least 1 Mem. NTP&Hhld can exit TP&Has Y. Children&Oth Mem Present&  H. Working&  W. Working\\
\hline
All         &         5.2&        18.9&        75.9&        55.8&        25.3&        97.2&        91.3\\
Has Y. Children&         7.3&        27.2&        65.5&       100.0&        17.9&        97.3&        89.3\\
No Y. Children&         2.6&         8.3&        89.1&         0.0&        34.6&        97.2&        93.9\\
Other H Member&         0.1&         2.6&        97.3&        39.5&       100.0&        96.4&        88.9\\
No Other Member&         7.0&        24.4&        68.7&        61.3&         0.0&        97.5&        92.1\\
Wife Works  &         5.7&        20.4&        73.9&        54.5&        24.7&        97.1&       100.0\\
Wife Not Working&         0.4&         2.8&        96.9&        69.0&        32.2&        98.3&         0.0\\
\hline\hline
\end{tabular}


}

}

\end{table}%

\section{Redistribution Scenarios}\label{redistribution-scenarios}

The idea of redistribution of household production responsibilities
follows the principle that everyone in a household should be able to
carry out their \textbf{fair} share of household work. But what
constitutes a fair share? In this section, we present three different
principles that could guide the redistribution of household production
responsibilities among eligible household members.

First, we use the simple egalitarianism principle that involves an equal
division of total household production time among all working-age
members. Second, we redistribute responsibilities based on the time
available to household members. In the third scenario, redistribution is
guided based on the principle of opportunity cost of time, where those
with higher wages (higher opportunity cost of time) are assigned less
household production time.

For all scenarios, we only consider the redistribution of required
household production activities \(R_j\) net of the portion met by
household members that are either disabled or are not part of the
working-age population. Thus, the goal is to simulate different
\(\alpha_{ij}\) values, which represent the share of required household
production time that each household member takes on. We also impose the
assumption that all household members are equally efficient at taking
care of the household responsibilities. We outline the methods used for
implementing the scenarios below.

\subsection{Scenario 1: Equal Shares}\label{scenario-1-equal-shares}

The first scenario considers the impact of redistributing household
production such that all working-age members of the household are
assigned an equal share of the required household production time. The
new share is defined as:

\begin{equation}\phantomsection\label{eq-r1}{\alpha_{ij}^E= \frac{1}{I_j}*(1-\alpha_{j}^{nw})
}\end{equation}

where \(\alpha_{ij}^E\) represents the redistributed share of individual
\(i\); \(I^j\) is the number of working-age persons in household \(j\)
and \(\alpha_{j}^{nw}\) represents the total share of all non-working
age household members. While this principle aligns with the idea of
equality, it overlooks time equity by redistributing tasks without
taking into consideration the time available to individuals.

\subsection{Scenario 2: Time Available}\label{scenario-2-time-available}

The time available scenario is based on the principles of equity. In
contrast with Scenario 1, this one suggests that household
responsibilities could be redistributed relative to the available time
individuals may have after setting aside the time for personal
maintenance requirements and income generation
(\(Z_{ij}=168-M-D_{ij}(L_{ij}+T_{ij})\)).

To implement this, we first calculate the time available (\(Z_{ij}\))
for each individual and recalculate the shares \(\alpha_{ij}^A\) using
the ratio of time available to the total time available among
working-age members. For individuals that do not have any time available
(\(Z_{ij}<0\)), we set their \(Z_{ij}\) to zero. This ensures that
people who already suffer from time poverty are not assigned further
tasks within the household. The new share is defined as:

\begin{equation}\phantomsection\label{eq-r2}{
\begin{aligned}
Z_{ij} &=\max{\Big(168-M-D_{ij}(L_{ij}+T_{ij}-E_{ij}(S_{ij}),0\Big)} \\
\alpha_{ij}^A &= \frac{Z_{ij}}{\sum Z_{ij}} (1-\alpha_{j}^{nw})
\end{aligned}
}\end{equation}

Because there are individuals (young adults) who may still be in school,
the standard definition of \(Z_{ij}\) may not capture their true time
availability. To address this, we add a correction to time availability
for all individuals who declared attending school, subtracting from
their available time (\(Z_{ij}\)) the average number of hours people
spend in education activities per week (\(S_{ij}\)). This correction
does not affect the time balance used for the identification of the time
poor, only the estimation of time available and the adjusted shares
\(\alpha_{ij}^A\).

\subsection{Scenario 3: Opportunity
Cost}\label{scenario-3-opportunity-cost}

The third possibility is based on the idea of opportunity costs along
marginalist lines. The sharing rule depends on the earning potentials of
individuals, such that individuals with higher potential wages are
assigned a lower share of household production time. In principle, this
would encourage the most productive members of the household to spend
more time in paid work, while those with lower earning potentials would
take on more household production responsibilities.

For example, if there are only three working-age adults in a household,
and where the second member earns twice as much as the first, and the
third earns three times as much as the first, the shares of household
production would be 1/2, 1/3, and 1/6 respectively. To implement this
scenario, we first calculate the inverse of the wage of each individual
\(rw_{ij}\), and then calculate the share of household production time
as follows:

\begin{equation}\phantomsection\label{eq-r3}{\begin{aligned}
rw_{ij} &= \frac{1}{w_{ij}} \\
\alpha_{ij}^O &= \frac{rw_{ij}}{\sum rw_{ij}} (1-\alpha_{j}^{nw})
\end{aligned}
}\end{equation}

where \(w_{ij}\) is the wage of individual \(i\).

Because we do not observe wage data for non-working household members,
we use the potential/predicted wages for all working-age household
members. To do this, we use a two-step procedure. First, we predict
occupation and industry probabilities for all non-working individuals
using a multinomial logit model. Second, we estimate a maximum
likelihood Heckman selection model (Heckman, 1979) using the observed
and predicted probabilities of belonging to specific occupations and
industries, in addition to individual, household, and spouse demographic
characteristics. With this information, we predict wages based on the
model that corrects for sample selection and use those wages as proxies
for the opportunity cost of time \(w_{ij}\).

\section{Impact of Redistribution}\label{impact-of-redistribution}

We start by looking at the impact on time poverty. Table~\ref{tbl-tb2}
Panel 1 presents the share of married men and women experiencing time
poverty. As expected more married women experience time poverty compared
to married men (66\% vs.~44\%). With redistribution, time poverty can be
reduced for married men and married women across all three scenarios,
and by a greater margin for women compared to men. This in turn reduces
gender disparity for couples. In other words, there is potential to
redistribute household production away from couples to other members in
the household, improving the well-being of working age married couples.

In rows 2 and 3 of Table~\ref{tbl-tb2}, we present the transition rates
i.e entry to and exit from time poverty respectively. Those who are
non-time poor in the baseline, some of them enters time poverty in all
three scenarios, with more married men entering poverty compared to
women except for in scenario 3, where 16 \% women became time poor
compared to 13.7\% men. Looking at the transition out of time poverty
for those who were time poor in the baseline, we find that more women
exited time poverty compared to men. For example, in Scenario 1, 42\%
men exited poverty compared to 75\% women, whereas in scenario 2, 70 \%
men exited compared to 60 \% women and in Scenario 3, the exit rates
were similar for both men and women at about 60 \%.

We next move on to examining time poverty rates and transition rates by
household types. The classifocation of households based on the presence
of time poor and time-non poor individuals and on total time deficit in
relation to time surplus are critical factors driving if redistribution
can effectively reduce time poverty in a gender-equitable manner. In
Panel 2 of Table~\ref{tbl-tb2}, we present the poverty rates by
household type. Clearly household type III where there is a mix of time
poor and time non poor individuals such that the total surplus outweighs
the total time deficit, redistribution is most effective, particularly
in scenario 2. While there is no way to make a time poor housheold time
non-poor where all members are time poor, there is potential to rachieve
greater gender-equitable sharing.

\begin{table}

\caption{\label{tbl-tb2}Time Poverty and Transition Rates}

\centering{

\begin{tabular}{l*{8}{c}cccccccccccccccc}
\hline\hline
          & \multicolumn{4}{c}{Men} & \multicolumn{4}{c}{Wife} \\ 
\cline{2-5} \cline{6-9}
            &    BL&  S. 1&  S. 2&  S. 3&    BL&  S. 1&  S. 2&  S. 3\\
\hline
All         &        43.8&        38.7&        23.8&        26.2&        61.0&        18.6&        22.2&        31.5\\
BL: Time NP &         0.0&        23.2&        19.3&        13.7&         0.0&         6.6&        15.6&        16.3\\
BL: Time P  &       100.0&        58.5&        29.6&        42.1&       100.0&        26.3&        26.4&        41.2\\
\midrule

Household Type      & \multicolumn{8}{c}{} \\ 
All Mem. TP         &       100.0&        95.8&        98.3&        82.4&       100.0&        68.7&        97.4&        81.8\\
At Least 1 Mem. NTP &        40.1&        82.4&        82.8&        55.8&        61.0&        53.5&        80.6&        65.1\\
Hhld can exit TP    &        40.9&        23.9&         4.1&        14.9&        58.4&         6.5&         2.4&        19.7\\
\hline\hline
\end{tabular}

}

\end{table}%

Perhaps add a table with all changes in time poverty for each scenario
and sub groups. Or do 3 , one for each scenario.

We now look at how these time poverty rates changes by household
structure. In Table~\ref{tbl-tb3}, we present the time poverty rates by
presence of children and presence of other members, both of which are
critical drivers of couples' time poverty incidence. Presence of young
household children would demand more time to be spent on hh production,
while presence of other members can off shoulder some of the hh
production work from couples. While presence of children is expected to
increase time poverty the latter is expected to decrease time poverty
for couples. At the baseline 63\% women experience time poverty compared
to 44 \% men when children are present, and in absence of children, time
poverty is similar for men 43\% and lower for women at 59\%. In both
cases share of time poor women is greater. With redistribution, time
poverty reduces among married women by a greater margin compared to
married men.

Moreover, when other members are present in the hh time poverty is lower
compared to when no other member is present. All three-redistribution
scenario reduces time poverty, particularly being more effective in
scenario 2 and when other members are present.

\begin{table}

\caption{\label{tbl-tb3}Time Poverty by Household Structure}

\centering{

\begin{tabular}{l*{8}{c}cccccccccccccccc}
\hline\hline
          & \multicolumn{4}{c}{Men} & \multicolumn{4}{c}{Wife} \\  \cline{2-5} \cline{6-9}
            &    BL&  S. 1&  S. 2&  S. 3&    BL&  S. 1&  S. 2&  S. 3\\
Yng Children Presence & \multicolumn{8}{c}{} \\ 
\ \ No Children&        43.1&        24.1&        13.0&        17.0&        59.0&        11.4&        11.4&        19.6\\
\ \ With Children&        44.4&        50.2&        32.5&        33.4&        62.6&        24.4&        30.7&        41.0\\
\midrule

Other Members in HH    & \multicolumn{8}{c}{} \\ 
\ \ No      &        44.4&        43.8&        29.4&        31.8&        62.0&        21.9&        27.5&        39.1\\
\ \ Yes     &        41.9&        23.6&         7.5&         9.6&        58.1&         9.0&         6.3&         9.3\\
\hline\hline
\end{tabular}

}

\end{table}%

Adjusted Income Poverty In Table~\ref{tbl-tb4}, we look at the time
poverty rate by poverty groups and employment/ earning status of wife.
We find that poverty rates are higher among married women compared to
married men across all poverty groups. Gender disparity is lowest among
below poverty group and increases for income-poverty ratio groups 1-2 \%
and 2-4\% bands before declining for over 4\% band.\\
Redistribution across all three scenarios reduces time poverty, more so
for women and most effectively in scenario 1 followed by scenarios 2 and
3.

In Panel 2 of Table~\ref{tbl-tb4}, interestingly we find that when wife
is working time poverty among men is much lower compared to women (40\%
vs.~66\%). With redistribution , time poverty decrease for both, thereby
also reducing gender disparity between couples, except for in scenario 3
where even after redistribution 35\% women experience time poverty
compared to 24\% men. This in turn indicates the vicious cycle exerted
by labor market inequality reflected in earnings which then translates
into assigning greater share of household production to women given that
they earn lower wages, which then also affect their labor market
participation and occupational segregation.

\begin{table}

\caption{\label{tbl-tb4}Time Poverty by Income}

\centering{

\begin{tabular}{l*{8}{c}cccccccccccccccc}
\hline\hline
          & \multicolumn{4}{c}{Men} & \multicolumn{4}{c}{Wife} \\  \cline{2-5} \cline{6-9}
            &    BL&  S. 1&  S. 2&  S. 3&    BL&  S. 1&  S. 2&  S. 3\\
Income/Pline   & \multicolumn{8}{c}{} \\ 
\ \ < PLine &        44.7&        35.3&        12.1&        24.3&        55.9&        10.8&        11.1&        20.5\\
\ \ 1-2 x Pline&        42.0&        39.4&        19.8&        25.4&        59.9&        16.8&        17.7&        28.6\\
\ \ 2-4 x Pline&        42.9&        38.2&        24.7&        26.1&        61.7&        18.7&        22.9&        31.1\\
\ \ >4 x Pline&        45.9&        39.1&        26.1&        26.8&        61.3&        20.4&        24.7&        34.7\\
\midrule

Wife Work Status   & \multicolumn{8}{c}{} \\ 
\ \ Not Working&        83.3&        65.9&         8.6&        48.9&         9.9&         0.0&         2.5&         0.1\\
\ \ Working &        40.0&        36.1&        25.3&        24.0&        65.9&        20.4&        24.0&        34.5\\
\hline\hline
\end{tabular}

}

\end{table}%

Statistics on the hidden poor Finally, in Table~\ref{tbl-tb5}, we look
at share of hidden poor, i.e share of housheolds who are not counted as
poor according to the official poverty line but are classified as poor
based on LIMTIP calculations because of individuals' time poverty. At
the baseline, 4.7 \% hh are hidden poor. With redistribution we can
decrease the share of hidden poor and make them visible in poverty
alleviation programs. Scenario 2 is most effective in reducing the share
of hidden poor. When we look at the share of hidden poor by household
types, we find that that the scope of reducing hidden poor is negligible
in scenario 1 where everyone is time poor and greatest in scenario 3
where time surplus exceeds time deficits, giving room for more effective
redistribution and to the extent of lifiting housheolds out of poverty.
Moreover, scenario 2 is the most effective in reducing the share of
hidden poor. Next, we look at the share of hidden poor if children are
present. Hidden poverty rate is higher when children are present
compared to in the absence of children (6.6\% vs.~2.5\%). This aligns
with higher time poverty rates for married couple with children. These
finding are crucial indicators of the need for publicly provided
childcare services that can relieve households of some of these
responsibilities and reducing the incidence of time-adjusted income
poverty. We also find that intra-hh redistribution can be effective in
reducing hidden poverty , particlulary for hosueholds with children and
in scenario 2.

\begin{table}

\caption{\label{tbl-tb5}Hidden Poor by Characteristics}

\centering{

\begin{tabular}{l*{4}{c}cccccccc}
\hline\hline
            &    Baseline&  Scenario 1&  Scenario 2&  Scenario 3\\
\hline
All         &         4.7&         1.8&         0.7&         1.8\\
\midrule

Household Type      & \multicolumn{4}{c}{} \\ 
\ \ All Mem. TP         &         5.3&         5.9&         5.7&         5.8\\
\ \ At Least 1 Mem. NTP &         6.9&         3.2&         1.8&         3.4\\
\ \ Hhld can exit TP    &         4.1&         1.2&         0.1&         1.1\\
\midrule

Yng Children Presence & \multicolumn{4}{c}{} \\ 
\ \ No Children &         2.5&         0.7&         0.2&         0.7\\
\ \ With Children &         6.6&         2.7&         1.1&         2.6\\
\midrule

Other Members in HH      & \multicolumn{4}{c}{} \\ 
\ \ No      &         4.5&         2.1&         0.9&         2.1\\
\ \ Yes     &         5.5&         1.1&         0.2&         0.9\\
\midrule

Income/Pline     & \multicolumn{4}{c}{} \\ 
\ \ 1-2 x Pline&        22.9&         8.6&         3.3&         8.3\\
\ \ 2-4 x Pline&         0.3&         0.2&         0.1&         0.2\\
\midrule

Wife Work Status  & \multicolumn{4}{c}{} \\ 
\ \ Not Working        &         7.8&         7.6&         0.5&         5.1\\
\ \ Working &         4.5&         1.3&         0.7&         1.5\\
\hline\hline
\end{tabular}

}

\end{table}%

\section{Policy Implications: Opportunities and
Challenges}\label{policy-implications-opportunities-and-challenges}

Redistribution can reduce time poverty, but only so much

Which is more effective? Are the results consistent with expectations?

What about based on household characteristics?

\section{Conclusion/recommendations}\label{conclusionrecommendations}

This policy brief has examined the potential of redistributing household
production responsibilities to alleviate time poverty in the United
States. Using the Levy Institute Measure of Time and Income Poverty
(LIMTIP), we have shown that time poverty is a significant issue
affecting 38.7\% of individuals living in time-poor households. Our
analysis of three redistribution scenarios - based on equality, equity,
and opportunity cost principles - reveals that such redistributions can
significantly reduce time poverty, particularly in households where time
surpluses exceed time deficits.

These findings underscore the importance of considering time poverty in
poverty alleviation efforts. They also highlight the potential of
intra-household redistribution as a policy tool to promote gender
equality and improve overall household well-being. However, the varying
effects across household types, hosuheold structures, poverty groups and
employment status of wives, along with variations across redistribution
scenarios suggest that a one-size-fits-all approach may not be optimal
and a targetted tailored approach is needed.

In conclusion, while redistribution of household production is promising
in alleviating time poverty, and the hidden poor, it should be
considered as part of strategies that also addresses societal and
structural factors that contribute to time and income poverty.

\phantomsection\label{refs}
\begin{CSLReferences}{1}{0}
\bibitem[\citeproctext]{ref-addati2018}
Addati, L., Cattaneo, U., Esquivel, V., and Valarino, I. (2018).
\emph{Care work and care jobs for the future of decent work}.
International Labour Organisation (ILO).

\bibitem[\citeproctext]{ref-Antonopoulos2017}
Antonopoulos, R., Esquivel, V., Masterson, T., and Zacharias, A. (2017).
Time and income poverty in the city of buenos aires. In R. Connelly and
E. Kongar (Eds.), \emph{Gender and time use in a global context: The
economics of employment and unpaid labor} (pp. 161--192). Palgrave
Macmillan US. \url{https://doi.org/10.1057/978-1-137-56837-3_7}

\bibitem[\citeproctext]{ref-berik2009}
Berik, G., Rodgers, Y. van der M., and Seguino, S. (2009). Feminist
{Economics} of {Inequality}, {Development}, and {Growth}. \emph{Feminist
Economics}, \emph{15}(3), 1--33.
\url{https://ezprox.bard.edu/login?url=https://search.ebscohost.com/login.aspx?direct=true&db=ecn&AN=1063369&site=eds-live&scope=site}

\bibitem[\citeproctext]{ref-hundt1996}
Bruyn-Hundt, M. (1996). Scenarios for a redistribution of unpaid work in
the netherlands. \emph{Feminist Economics}, \emph{2}(3), 129--133.
\url{https://doi.org/10.1080/13545709610001707826}

\bibitem[\citeproctext]{ref-duflo2012}
Duflo, E. (2012). Women {Empowerment} and {Economic} {Development}.
\emph{Journal of Economic Literature}, \emph{50}(4), 1051--1079.
\url{https://doi.org/10.1257/jel.50.4.1051}

\bibitem[\citeproctext]{ref-elson2017}
Elso, D. (2017). \emph{Recognize, {Reduce}, and {Redistribute} {Unpaid}
{Care} {Work}: {How} to {Close} the {Gender} {Gap}}.
\url{https://doi.org/10.1177/1095796017700135}

\bibitem[\citeproctext]{ref-elson2009}
Elson, D. (2009). Gender {Equality} and {Economic} {Growth} in the
{World} {Bank} {World} {Development} {Report} 2006. \emph{Feminist
Economics}, \emph{15}(3), 35--59.
\url{https://doi.org/10.1080/13545700902964303}

\bibitem[\citeproctext]{ref-valeria2016}
Esquivel, V. (2016). Power and the {Sustainable} {Development} {Goals}:
A feminist analysis. \emph{Gender \& Development}, \emph{24}(1), 9--23.
\url{https://doi.org/10.1080/13552074.2016.1147872}

\bibitem[\citeproctext]{ref-heckman1979}
Heckman, J. J. (1979). Sample selection bias as a specification error.
\emph{Econometrica}, \emph{47}(1), 153.
\url{https://doi.org/10.2307/1912352}

\bibitem[\citeproctext]{ref-hess2020}
Hess, Cynthia, Ahmed, T., and Hayes, J. (2020). \emph{Providing {Unpaid}
{Household} and {Care} {Work} in the {United} {States}: {Uncovering}
{Inequality}}.

\bibitem[\citeproctext]{ref-malik2018}
Malik, R., Hamm, K., Schochet, L., Novoa, C., Workman, S., and
Jessen-Howard, S. (2018). America's {Child} {Care} {Deserts} in 2018.
\emph{Center for American Progress}.
\url{https://www.americanprogress.org/article/americas-child-care-deserts-2018/}

\bibitem[\citeproctext]{ref-masterson2012}
Masterson, T. (2012). \emph{Simulations of full-time employment and
household work in the levy institute measure of time and income poverty
(LIMTIP) for argentina, chile, and mexico} (Working Paper 727; Levy
Economics Institute Working Paper). Levy Economics Institute of Bard
College.

\bibitem[\citeproctext]{ref-masterspm2022}
Masterson, T., Antonopoulos, R., Nassif-Pires, L., Rios-Avila, F., and
Zacharias, A. (2022). \emph{Assessing the impact of childcare expansion
in mexico: Time use, employment, and poverty} {[}Research Project
Report{]}. Levy Economics Institute of Bard College.
\url{https://www.levyinstitute.org/pubs/rpr_6_22.pdf}

\bibitem[\citeproctext]{ref-oecd2020}
OECD. (2020). \emph{Early learning and child well-being in the united
states} (p. 124).
https://doi.org/\url{https://doi.org/https://doi.org/10.1787/198d8c99-en}

\bibitem[\citeproctext]{ref-rein2023}
Reinhard, S. C., Caldera, S., Houser, A., and Choula, R. (2023).
\emph{Valuing the {Invaluable}: 2023 {Update}}. AARP Public Policy
Institute. \url{https://doi.org/10.26419/ppi.00082.006}

\bibitem[\citeproctext]{ref-zacharias2011}
Zacharias, A. (2011). \emph{The measurement of time and income poverty}
(Working Paper 690; Levy Economics Institute Working Paper). Levy
Economics Institute of Bard College.

\bibitem[\citeproctext]{ref-zacharias2012}
Zacharias, A., Antonopoulos, R., and Masterson, T. (2012). \emph{Why
time deficits matter: Implications for the measurement of poverty}
{[}Research Project Report{]}. Levy Economics Institute of Bard College.
\url{https://www.levyinstitute.org/pubs/rpr_08_12}

\bibitem[\citeproctext]{ref-zacharias2014}
Zacharias, A., Masterson, T., and Kim, K. (2014). \emph{The measurement
of time and income poverty in korea: The levy institute measure of time
and income poverty} {[}Research Project Report{]}. Levy Economics
Institute of Bard College.
\url{https://www.levyinstitute.org/pubs/rpr_8_14}

\bibitem[\citeproctext]{ref-zacharias2018}
Zacharias, A., Masterson, T., Rios-Avila, F., Kim, K., and
Khitarishvili, T. (2018). \emph{The measurement of time and income
poverty in ghana and tanzania: The levy institute measure of time and
consumption poverty} {[}Research Project Report{]}. Levy Economics
Institute of Bard College.
\url{https://www.levyinstitute.org/pubs/rpr_8_18}

\bibitem[\citeproctext]{ref-zacharias2021}
Zacharias, A., Masterson, T., Rios-Avila, F., and Oduro, A. D. (2021).
\emph{Scope and effects of reducing time deficits via intrahousehold
redistribution of household production: Evidence from sub-saharan
africa: The levy institute measure of time and consumption poverty}
{[}Research Project Report{]}. Levy Economics Institute of Bard
College.\href{\%20https://www.levyinstitute.org/pubs/rpr_7_21.pdf}{https://www.levyinstitute.org/pubs/rpr\_7\_21.pdf}

\bibitem[\citeproctext]{ref-zacharias2024a}
Zacharias, A., Rios-Avila, F., Folbre, N., and Masterson, T. (2024).
\emph{Integrating nonmarket consumption into the bureau of labor
statistics consumer expenditure survey} {[}Research Project Report{]}.
Levy Economics Institute of Bard
College.\href{\%20https://www.bls.gov/cex/consumption/integrating-nonmarket-consumption-bls-consumer-expenditure-survey.pdf}{https://www.bls.gov/cex/consumption/integrating-nonmarket-consumption-bls-consumer-expenditure-survey.pdf}

\end{CSLReferences}



\end{document}
